\documentclass[titlepage]{scrartcl}
\usepackage[ngerman]{babel}
\usepackage[T1]{fontenc}
\usepackage[utf8]{inputenc}
\usepackage{graphics}

\titlehead{Wintersemester 2023/24}
\title{\TicTacToe}
\subtitle{Product Backlog}
\date{Stand: \today}
\author{Jonas, Luis, Leonid}

\newcommand{\TicTacToe}{TI\reflectbox K Tac Toe}

\begin{document}
\maketitle

%%%Aufgabenverteilung: 
%Produkt-Ziel: Leonid
%Funktionalitäten: Luis (Block1), Jonas (Block2)
%DoD: Da müssen wir uns teilweise einigen, aber vielleicht macht auch jemand Vorschläge
%DoF: aufschieben
%Scrum-Master: aufschieben
%
%Paket

\section{Produkt-Ziel}%schriftlicher Pitch der Software
Künstliche Intelligenz, insbesondere künstliche neuronale Netze, sind eines der Hot Topics in der Informatik.
Nicht zuletzt mit dem Aufkommen von Large Language Models muss im Jahr 2023 überall KI rein, wo KI irgendwie Platz hat.
Auch im Bildungsplan Informatik in Baden-Württemberg kommt das Thema künstliche Intelligenz wahrscheinlich irgendwann in den nächsten 100 Jahren an.

Die App "`\TicTacToe"' soll einen Beitrag für den sanften Einstieg in das Thema maschinelles Lernen leisten.
Schülerinnen und Schüler können mit wenigen Klicks einem regelbasiertem Expertensystem zuschauen, das nach und nach das Spiel "`Tic Tac Toe"' erlernt.
Dabei erfahren sie in diesem einfachen Setting spielerisch das Konzept eines Brute-Force-Ansatzes sowie der Fehlerrückführung.

Die geplanten Funktionalitäten erlauben einen Einsatz im Unterricht nach einer Heranführung durch die Lehrkraft.
Wir gehen davon aus, dass den Schülerinnen und Schülern die Hexapawn-Variante HER bereits bekannt ist.
Im Webinterface von "`\TicTacToe"' können die SuS den Entscheidungsbaum der KI erkunden, selbst gegen sie spielen oder zwei KIs gegeneinander trainieren lassen.



%Erklären warum unsere Software cool ist (Stichwort: Einsatz)
\section{Funktionalitäten}% Kriterien: (vollständige) Definition der Software
\subsection{Muss-Kriterien}
	%Block1
	\begin{itemize}
		\item[M100] Ein 3x3-TicTacToe-Spielfeld wird angezeigt.
		\item[M200] Die Felder des Spielfeldes können genau nach den Regeln von Tic-Tac-Toe besetzt werden.
		\item[M300] Die vorherigen Konfigurationen des Spielfeldes können in einem History-Stack gespeichert und in einem Baum angezeigt werden.
		\item[M400] Alle Spielbrett-Konfigurationen, welche auf die momentane folgen können, können angezeigt werden, bis auf semantisch äquivalente Situationen.
		\item[M600] Eine einfache KI kann auf Basis von Gewichten einen Zug auswählen.
		\item[M700] Die KI kann ein- und ausgeschaltet werden.
		\item[M800] Um die KI zu trainieren können durch veschiedene Belohnungsstrategien die Gewichte der Pfade verändert werden.
	\end{itemize}
	%Tic-Tac-Toe-Brett und Regeln
	%History-Stack
	%Visualisierung der nachfolgenden Konfigurationen
	%Visualisierung einer einfachen KI (Elimination)
	%Block2
	%Ein- und Ausschalten von KI-Zügen
	%Spielereinfluss auf Gewichte?
	\subsection{Soll-Kriterien}
	\begin{itemize}
		\item[S100] Es gibt die Möglichkeit, zwei KIs als Kontrahenten auszuwählen. Sie trainnieren sich so gegenseitig.
		\item[S200] Die Zuggeschwindigkeit der KI kann festgelegt werden.
		\item[S300] Der Nutzer kann einstellen, dass nach Beenden eines Spiels automatisch das nächste Spiel gestartet wird.
		\item[S400] Der Nutzer kann einstellen, dass nach beenden eines Spiels die Start-Reihenfolge der Spieler getauscht wird.
		\item[S500] Es gibt zu jedem Zeitpunkt die Möglichkeit, das aktuelle Spiel abzubrechen. Dabei werden die Gewichte der aktuell trainnierten KI nicht gelöscht.
		\item[S600] Falls das Spiel abgebrochen wurde und die KI-Gewichte noch nicht gelöscht wurden, so besteht dann die Möglichkeit die Gewichte der trainnierte(n) KI(s) jeweils einzeln zu reseten.
		\item[S700] Die Belohnungsstrategien können am Ende eines Spiels vom Nutzer angepasst werden. Zur Auswahl stehen Elimination des letzten Zuges bei Verlust des Spiels, eine Umgewichtung des letzten Zuges (wobei hier die Umgewichtung jeweils bei Gewinnen, Unentschieden und Verlust einzeln festgelegt werden können) und eine vereinfachte Version der Backpropagation (hier wird jeweils für Gewinnen, Unentschieden und Verlust für jeden einzelnen gemachten Zug einzeln das Gewicht um den vom Nutzer eingestellten Wert verändert).
		\item[S800] Falls nicht die Option aktiviert ist, dass am Ende eines Spiels direkt das nächste Spiel gestartet wird, wird der Pfad des Spiels mit seinen Zwischenständen dargestellt!!!!!!!!!!!!!!!!!!!!!!!!!!!!!!!!!!!!!!!!!!!!!!!!!!!!!!!.
	\end{itemize}

\subsection{Kann-Kriterien}
	\begin{itemize}
		\item[K100] Es ist eine fehlerlose KI implementiert. Bei Start des Spiels kann diese fehlerlose KI anstatt eines Spielers oder einer anderen KI ausgewählt werden.
		\item[K200] Es gibt solange das Spiel still steht die Möglichkeit, die aktuell trainnierte(n) KI(s) zu exportieren. Die exportierten Gewichte können solange das Spiel still steht, importiert werden und überschreiben dann die aktuellen Gewichte der ausgewählten KI.
		\item[K300] Die Entscheidungsbaum-Struktur ist so aufgebaut, dass das Tic Tac Toe Spiel durch ein anderes einfaches Brettspiel ersetzt werden kann und die Entscheidungsbaum-Struktur weiterhin ihren Zweck erfüllt.
	\end{itemize}
	%%%Spätestens hier beginnt Wunschdenken
	%Auswahl mehrerer KIs
	%Implementierung von komplexerer KI (gewichtet Pfad)
	%Implementierung von fehlerloser KI
	%automatisches Trainieren
	%%%Absolutes Wunschdenken
	%Interface für andere Spiele als Tic-Tac-Toe anbieten
	
\subsection{Abgrenzungskriterien}
	\begin{itemize}
		\item[A100] Die Gewichte der KI können nicht manuell angepasst werden
	\end{itemize}
	
\section{Definition of Done}%Checkliste zur Überprüfung, ob ein Increment fertig ist
\begin{itemize}
	\item UnitTests aller implementierten Features laufen durch
	\item Features des Increments ist in den \texttt{main}-Branch gemergt
	\item Wichtige Funktionalitäten der neuen Features sind mit UnitTests versehen
	\item \texttt{run dev} läuft durch
	\item Trello-Tickets sind auf "`Abgeschlossen"'
\end{itemize}

\section{Definition of Fun}%Vereinbarungen über Zusammenarbeit
Wir vereinbaren die folgenden Regeln für unsere Zusammenarbeit:
\begin{itemize}
	\item Wir treffen uns vor jedem Dienstagstreffen 15 min vorher zur Koordination
	\item Wir treffen uns jede Woche Freitags um 08:00 Uhr via Discord
	\item Zu unseren Treffen bemühen wir uns pünktlich und vorbereitet zu kommen
	\item Wenn wir auf Probleme stoßen (Implementierung, Setup, Zeitprobleme...), so kommunizieren wir diese offen
	\item Wir dokumentieren unseren Fortschritt auf Trello
	\item Eine Änderung am produktiven Code wird nur in den Mainbranch gemergt, wenn sie von einer zweiten Person gereviewt wurde.
\end{itemize}

%Vielleicht kann man hier einen typischen Arbeitsablauf skizzieren?
%Wollen wir commits im Discord erwähnen?
%Was erwarten wir, wie oft die anderen im Discord lesen?
\section{Wichtige Entscheidungen}
\begin{itemize}
\item Ein 3x3-Spielbrett wird auf einer HTML und Javascipt basierten Webseite angezeigt und auf diesem ist Tic-Tac-Toe spielbar.
\end{itemize}
%Ich glaube hier gibt es noch nicht so richtig was
\section{Projektplanung}%Verteilung der Scrum-Master-Tätigkeit
\begin{tabular}{lc}
	Sprint & Scrum-Master\\
	Entwurf & \\
	Quali-sicherung & \\
	Abschluss & \\
\end{tabular}
\end{document}

\section{Epic}
Epic über die Benutzung des TicTacToeamprojekts:

Ein Schüler öffnet die Website zum ersten Mal und sieht einen Bildschirm, auf dem für Spieler 1 (Sp1) und Spieler 2 
(Sp2) ausgewählt werden kann, ob eine KI oder ein Spieler den Zug übernimmt. Der Schüler wählt aus, dass ein 
Spieler (Sp1) gegen eine KI (Sp2) spielt und startet das Spiel per Knopfdruck. Er setzt den ersten Stein auf ein 
Feld seiner Wahl und schaut sich an der Seite alle möglichen nächsten Züge von Sp2 an. Er drückt auf den 
„next“-Knopf, um den Sp2 ziehen zu lassen. Nach seinem nächsten Zug aktiviert er den „Start“-Knopf, um Sp2 direkt 
ziehen zu lassen. Nachdem das Spiel fertiggespielt ist, trägt er den von ihm gewünschten Belohnungsalgorithmus 
in das aufgekommene Fenster ein und startet das Spiel per Knopfdruck neu, wodurch er wieder auf den ersten Bildschirm 
kommt. Er entscheidet sich weiterhin gegen die gleiche KI zu spielen, um sie zu trainieren und schaltet dafür ein, 
dass der erste Bildschirm nicht wieder angezeigt wird. Da er auch mit dem Belohnungsalgorithmus zufrieden ist stellt 
er auch ein, dass der Belohnungsbildschirm nicht mehr angezeigt wird und erhöht die Geschwindigkeit, in der die Sp2 
einen Zug macht. Nachdem er die KI nach seinen Wünschen trainiert hat stellt er ein, dass er wieder in den Startbildschirm 
kommt und löscht den Lernfortschritt der KI um sie diesmal mit einem anderen Belohnungsalgorithmus zu trainieren.

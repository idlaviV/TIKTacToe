\documentclass[titlepage]{scrartcl}
\usepackage[ngerman]{babel}
\usepackage[T1]{fontenc}
\usepackage[utf8]{inputenc}

\titlehead{Wintersemester 2023/24}
\title{Tic Tac Toe KI Web}
\subtitle{Product Backlog}
\date{Stand: \today}
\author{Jonas, Luis, Leonid}

\begin{document}
\maketitle

%%%Aufgabenverteilung: 
%Produkt-Ziel: Leonid
%Funktionalitäten: Luis (Block1), Jonas (Block2)
%DoD: Da müssen wir uns teilweise einigen, aber vielleicht macht auch jemand Vorschläge
%DoF: aufschieben
%Scrum-Master: aufschieben
%
%Paket

\section{Produkt-Ziel}%schriftlicher Pitch der Software
tba
%Erklären warum unsere Software cool ist (Stichwort: Einsatz)
\section{Funktionalitäten}% Kriterien: (vollständige) Definition der Software
tba
%Block1
%Tic-Tac-Toe-Brett und Regeln
%History-Stack
%Visualisierung der nachfolgenden Konfigurationen
%Visualisierung einer einfachen KI (Elimination)
%Block2
%Ein- und Ausschalten von KI-Zügen
%Spielereinfluss auf Gewichte?
%%%Spätestens hier beginnt Wunschdenken
%Auswahl mehrerer KIs
%Implementierung von komplexerer KI (gewichtet Pfad)
%Implementierung von fehlerloser KI
%automatisches Trainieren
%%%Absolutes Wunschdenken
%Interface für andere Spiele als Tic-Tac-Toe anbieten
\section{Definition of Done}%Checkliste zur Überprüfung, ob ein Increment fertig ist
tba
%%%Beispiele
%UnitTests laufen
%run dev läuft
%Frage: Haben wir einen Ablauf um Funktionalität zu testen / ob oder welche Ziele erreicht wurden?
%Idee: Codeänderungen nur auf Branches pushen, jemand anderes macht Review und mergt dann
%Gibt es ein Konzept wann Tests geschrieben werden?
%Trello-Todos sind fertig
\section{Definition of Fun}%Vereinbarungen über Zusammenarbeit
tba
%Probleme nicht verstecken
%Pünktlichkeit bei Treffen?
%Abgrenzung zu DoD ist mir nicht klar
%Vielleicht kann man hier einen typischen Arbeitsablauf skizzieren?
%Wollen wir commits im Discord erwähnen?
%Was erwarten wir, wie oft die anderen im Discord lesen?
\section{Wichtige Entscheidungen}
tba
%Ich glaube hier gibt es noch nicht so richtig was
\section{Projektplanung}%Verteilung der Scrum-Master-Tätigkeit
\begin{tabular}{lc}
	Sprint & Scrum-Master\\
	Entwurf & \\
	Quali-sicherung & \\
	Abschluss & \\
\end{tabular}
\end{document}
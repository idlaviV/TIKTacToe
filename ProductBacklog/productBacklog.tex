\documentclass[titlepage]{scrartcl}
\usepackage[ngerman]{babel}
\usepackage[T1]{fontenc}
\usepackage[utf8]{inputenc}
\usepackage{graphics}

\titlehead{Wintersemester 2023/24}
\title{\TicTacToe}
\subtitle{Product Backlog}
\date{Stand: \today}
\author{Jonas, Luis, Leonid}

\newcommand{\TicTacToe}{TI\reflectbox K Tac Toe}

\begin{document}
\maketitle

\emph{Hinweis:} Aufgrund der besseren Lesbarkeit ist in diesem Dokument nur von Spielern und Nutzern die Rede.
Es sind aber immer Spielerinnen und Spieler und Nutzerinnen und Nutzer aller Geschlechter gemeint.

"`Spieler"' meint im Folgenden immer die Entität, die eine der beiden Positionen im Spiel Tic-Tac-Toe (Kreuz oder Kreis) übernommen hat.
Die Person, die die App benutzt, wird "`Nutzer"' genannt.

\section{Produkt-Ziel}%schriftlicher Pitch der Software
Künstliche Intelligenz, insbesondere künstliche neuronale Netze, sind eines der Hot Topics in der Informatik.
Nicht zuletzt mit dem Aufkommen von Large Language Models muss im Jahr 2023 überall KI rein, wo KI irgendwie Platz hat.
Auch im Bildungsplan Informatik in Baden-Württemberg kommt das Thema künstliche Intelligenz wahrscheinlich irgendwann in den nächsten 100 Jahren an.

Die App "`\TicTacToe"' soll einen Beitrag für den sanften Einstieg in das Thema maschinelles Lernen leisten.
Schülerinnen und Schüler können mit wenigen Klicks einem regelbasiertem Expertensystem zuschauen, das nach und nach das Spiel "`Tic Tac Toe"' erlernt.
Dabei erfahren sie in diesem einfachen Setting spielerisch das Konzept eines Brute-Force-Ansatzes sowie der Fehlerrückführung.

Die geplanten Funktionalitäten erlauben einen Einsatz im Unterricht nach einer Heranführung durch die Lehrkraft.
Wir gehen davon aus, dass den Schülerinnen und Schülern die Hexapawn-Variante HER bereits bekannt ist.
Im Webinterface von "`\TicTacToe"' können die SuS den Entscheidungsbaum der KI erkunden, selbst gegen sie spielen oder zwei KIs gegeneinander trainieren lassen.



%Erklären warum unsere Software cool ist (Stichwort: Einsatz)
\section{Funktionalitäten}% Kriterien: (vollständige) Definition der Software
\subsection{Muss-Kriterien}
	%Block1
	\begin{itemize}
		\item[M100] Während des Spiels wird ein 3x3-Tic-Tac-Toe-Spielfeld angezeigt.
		\item[M200] Die Felder des Spielfeldes können genau nach den Regeln von Tic-Tac-Toe besetzt werden.
		\item[M300] Die vorherigen Konfigurationen des Spielfeldes werden gespeichert und in einem Baum visualisiert.
		\item[M310] Die bisher ausgeführten Züge können rückgängig gemacht werden.%%Neu Soll das später kommen?
		\item[M400] Alle Spielbrett-Konfigurationen, welche auf die momentane folgen können, können angezeigt werden (bis auf semantisch äquivalente Situationen).
		\item[M600] Eine einfache KI wählt auf Basis von Gewichten einen Zug aus.
		\item[M700] Um die KI zu trainieren werden durch Belohnungsstrategien die Gewichte der Pfade verändert.
		\item[M710] Es gibt mindestens eine Belohnungsstrategie.
		\item[M800] Der Nutzer kann die gegen die KI Tic-Tac-Toe spielen.%%Neu
	\end{itemize}


	\subsection{Soll-Kriterien}
	\begin{itemize}
		\item[S100] Automatisches Ziehen der KI kann an- und abgewählt werden.
		\item[S110] Ist das automatische Ziehen deaktiviert, so gibt es die Möglichkeit einzelne Züge der KI auszulösen.
		\item[S200] Es gibt die Möglichkeit, zwei KIs als Spieler auszuwählen.
		\item[S300] Die Zuggeschwindigkeit der KI kann festgelegt werden.
		\item[S400] Der Nutzer kann einstellen, dass nach Beenden eines Spiels automatisch das nächste Spiel mit den gleichen Einstellungen gestartet wird.
		\item[S500] Zwischen zwei Spielen besteht die Möglichkeit, die Gewichte der trainierte(n) KI(s) einzeln zurückzusetzen.
		\item[S600] Es gibt eine Ansicht, in der am Ende des Spiels die Belohnungsstrategie vom Nutzer angepasst werden kann.
		\item[S610] Eine Belohnungsstrategie ist die Elimination des letzten Zuges bei Verlust des Spiels.
		\item[S620] Eine Belohnungsstrategie ist Fehlerrückführung.
				Für Gewinnen, Unentschieden und Verlust werden die Gewichte der ausgeführten Züge individuell angepasst.
		\item[S630] Der Nutzer kann auswählen, dass die konfigurierte Belohnungsstrategie in Zukunft automatisch angewendet wird. %%Hat noch gefehlt
		\item[S700] Während der Auswahl der Belohnungsstrategie wird der Gewichte-Graph angezeigt.
		\item[S710] Der Spielverlauf wird im Gewichte-Graph hervorgehoben.
	\end{itemize}

\subsection{Kann-Kriterien}
	\begin{itemize}
		\item[K100] Bei der Belohnungsstrategie Fehlerrückführung ist eine Aufschlüsselung der Belohnungen nach Spielstadium möglich.
				Das heißt, dass die Gewichte in der frühen Phase des Spiels anders angepasst werden können als die in der späteren Phase.
		\item[K110] Bei der Konfiguration der Belohnungsstrategie kann der Nutzer auswählen, dass die KI auch die Gewichte der Züge des Gegners anpasst.%Neu
		\item[K200] Es ist eine Ansicht aufrufbar, in der der gesamte Gewichte-Graph angezeigt wird.
		\item[K300] Es ist eine perfekt spielende KI implementiert.
				Bei Start des Spiels kann diese fehlerlose KI anstelle des Nutzers oder einer anderen KI ausgewählt werden.
		\item[K400] Es gibt zu jedem Zeitpunkt die Möglichkeit, das aktuelle Spiel abzubrechen.
				Dabei werden die Gewichte der aktuell trainierten KI nicht geändert.
		\item[K500] Es gibt, solange das Spiel still steht, die Möglichkeit, die aktuell trainierte(n) KI(s) zu exportieren.
				Solange das Spiel still steht können Gewichte importiert werden, die dann die aktuellen Gewichte der ausgewählten KI überschreiben.
		\item[K600] Der Nutzer kann einstellen, dass nach Beenden eines Spiels die Start-Reihenfolge der Spieler automatisch getauscht wird.
		\item[K700] Statt Tic-Tac-Toe kann auch ein einfaches Brettspiel gespielt werden, für das die KI mit einer ähnlichen Entscheidungsbaum-Struktur trainiert wird.%Umformuliert, damit überprüfbar
		%\item[K700] Die Entscheidungsbaum-Struktur ist so aufgebaut, dass das Tic-Tac-Toe-Spiel durch ein anderes einfaches Brettspiel ersetzt werden kann und die Entscheidungsbaum-Struktur weiterhin ihren Zweck erfüllt.

	\end{itemize}

	
\subsection{Abgrenzungskriterien}
	\begin{itemize}
		\item[A100] Die Gewichte der KI können nur über vorgegebene Belohnungsstrategien angepasst werden.
		\item[A200] Die Entscheidungen der KI basieren nur auf der aktuellen Spielsituation, nicht auf dem Spielverlauf.
	\end{itemize}
	
\section{Definition of Done}%Checkliste zur Überprüfung, ob ein Increment fertig ist
Ein Increment besteht aus einem oder mehreren Features.
Falls folgende Eigenschaften auf unser Projekt zutreffen, erklären wir das Feature für abgeschlossen:
\begin{itemize}
	\item UnitTests aller bisher implementierten Features laufen durch.
	\item Der Code des Features ist in den \texttt{main}-Branch gemergt.
	\item Eine relevante Änderung am produktiven Code wird nur in den \texttt{main}-Branch gemergt, wenn sie von einer zweiten Person gereviewt wurde.
	\item Wichtige Funktionalitäten des Features sind mit UnitTests versehen.
	\item \texttt{run dev} läuft durch.
	\item Für das neue Feature wurde nach dem Merge ein Integrationstest durchgeführt.
	\item Trello-Tickets des Features haben den Status "`Abgeschlossen"'.
\end{itemize}

\section{Definition of Fun}%Vereinbarungen über Zusammenarbeit
Wir vereinbaren die folgenden Regeln für unsere Zusammenarbeit:
\begin{itemize}
	\item Wir treffen uns vor jedem Dienstagstreffen 15 min vorher zur Koordination.
	\item Wir treffen uns jede Woche Freitags um 08:00 Uhr via Discord.
	\item Zu unseren Treffen kommen wir pünktlich und vorbereitet.%Wenns Spaß machen soll sollten wir das auch tun
	\item Wenn wir auf Probleme stoßen (Implementierung, Setup, Zeitprobleme...), so kommunizieren wir diese offen und zeitnah.
	\item Wir dokumentieren unseren Fortschritt auf Trello.
	\item Wir behalten im Blick, dass neuer Code so früh abgeschlossen werden muss, dass ein Review noch vor der Abgabe möglich ist.
	\item Wir berichten im Discord über unseren Arbeitsfortschritt.
\end{itemize}

\section{Wichtige Entscheidungen}
\begin{itemize}
\item Ein 3x3-Spielbrett wird auf einer HTML und Javascipt basierten Webseite angezeigt und auf diesem ist Tic-Tac-Toe spielbar.
\end{itemize}

\section{Offene Fragen}
\begin{itemize}
	\item Wo werden die Gewichte im Baum visualisiert? An den Kanten oder Knoten?
	\item Kann der gesamte Graph der Gewichte überhaupt sinnvoll und kompakt visualisiert werden?
\end{itemize}


\section{Projektplanung}%Verteilung der Scrum-Master-Tätigkeit
\begin{tabular}{lcc}
	Sprint &Thema& Scrum-Master\\
	0&Anforderung&--\\
	1&Entwurf & Luis\\
	2&Quali-sicherung & Jonas \\
	3&Abschluss & Leonid\\
\end{tabular}


\section{Epic}
%Epic über die Benutzung des TicTacToeamprojekts:

Ein Schüler öffnet die Website zum ersten Mal und sieht den Startbildschirm.
Auf diesem kann für Spieler 1 (Sp1) und Spieler 2 (Sp2) ausgewählt werden kann, ob eine KI oder der Schüler die Züge übernimmt.
Der Schüler wählt aus, dass er selbst (Sp1) gegen eine KI (Sp2) spielt und startet das Spiel per Knopfdruck.
Er setzt den ersten Stein auf ein Feld seiner Wahl und schaut sich an der Seite alle möglichen nächsten Züge von Sp2 an.
Er drückt auf einen Knopf, damit die KI für Sp2 zieht.
Nach seinem nächsten Zug wählt er aus, dass die KI die Züge ab jetzt automatisch spielt.
Im Folgenden macht die KI ihre Züge jetzt sobald der Schüler für Sp1 gezogen hat.
Nachdem das Spiel fertig gespielt ist wählt er die Belohnungsstrategie aus.
Dadurch werden die Gewichte der KI entsprechend angepasst.
Dies kann der Schüler im angezeigten Gewichte-Graph nachvollziehen.

Dann startet er das Spiel per Knopfdruck neu, wodurch er wieder auf den Startbildschirm kommt.
Er entscheidet sich gegen die gleiche KI erneut zu spielen, um sie weiter zu trainieren.
Er schaltet ein, dass der Startbildschirm nicht wieder angezeigt wird.
Da er auch mit dem Belohnungsalgorithmus zufrieden ist stellt er auch ein, dass der Belohnungsbildschirm nicht mehr angezeigt wird und erhöht die Geschwindigkeit, in der die KI einen Zug macht.

Im Folgenden spielt der Schüler gegen die KI mehrere Spiele, ohne dass der Startbildschirm oder der Belohnungsbildschirm angezeigt werden.
Die Gewichte der KI werden trotzdem nach jedem Spiel angepasst.

Nachdem er die KI nach seinen Wünschen trainiert hat, stellt er ein, dass er wieder in den Startbildschirm kommt.
Er löscht den Lernfortschritt der KI, um sie diesmal mit einem anderen Belohnungsalgorithmus zu trainieren.

\end{document}

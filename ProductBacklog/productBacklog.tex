\documentclass[titlepage]{scrartcl}
\usepackage[ngerman]{babel}
\usepackage[T1]{fontenc}
\usepackage[utf8]{inputenc}
\usepackage{graphics}

\titlehead{Wintersemester 2023/24}
\title{\TicTacToe}
\subtitle{Product Backlog}
\date{Stand: \today}
\author{Jonas, Luis, Leonid}

\newcommand{\TicTacToe}{TI\reflectbox K Tac Toe}

\begin{document}
\maketitle

%%%Aufgabenverteilung: 
%Produkt-Ziel: Leonid
%Funktionalitäten: Luis (Block1), Jonas (Block2)
%DoD: Da müssen wir uns teilweise einigen, aber vielleicht macht auch jemand Vorschläge
%DoF: aufschieben
%Scrum-Master: aufschieben
%
%Paket

\section{Produkt-Ziel}%schriftlicher Pitch der Software
Künstliche Intelligenz, insbesondere künstliche neuronale Netze, sind eines der Hot Topics in der Informatik.
Nicht zuletzt mit dem Aufkommen von Large Language Models muss im Jahr 2023 überall KI rein, wo KI irgendwie Platz hat.
Auch im Bildungsplan Informatik in Baden-Württemberg kommt das Thema künstliche Intelligenz wahrscheinlich irgendwann in den nächsten 100 Jahren an.

Die App "`\TicTacToe"' soll einen Beitrag für den sanften Einstieg in das Thema maschinelles Lernen leisten.
Schülerinnen und Schüler können mit wenigen Klicks einem regelbasiertem Expertensystem zuschauen, das nach und nach das Spiel "`Tic Tac Toe"' erlernt.
Dabei erfahren sie in diesem einfachen Setting spielerisch das Konzept eines Brute-Force-Ansatzes sowie der Fehlerrückführung.

Die geplanten Funktionalitäten erlauben einen Einsatz im Unterricht nach einer Heranführung durch die Lehrkraft.
Wir gehen davon aus, dass den Schülerinnen und Schülern die Hexapawn-Variante HER bereits bekannt ist.
Im Webinterface von "`\TicTacToe"' können die SuS den Entscheidungsbaum der KI erkunden, selbst gegen sie spielen oder zwei KIs gegeneinander trainieren lassen.



%Erklären warum unsere Software cool ist (Stichwort: Einsatz)
\section{Funktionalitäten}% Kriterien: (vollständige) Definition der Software
\subsection{Muss-Kriterien}
	%Block1
	\begin{itemize}
		\item[M100] Ein 3x3-TicTacToe-Spielfeld wird angezeigt.
		\item[M200] Die Felder des Spielfeldes können genau nach den Regeln von Tic-Tac-Toe besetzt werden.
		\item[M300] Die vorherigen Konfigurationen des Spielfeldes können in einem History-Stack gespeichert und in einem Baum angezeigt werden.
		\item[M400] Alle Spielbrett-Konfigurationen, welche auf die momentane folgen können, können angezeigt werden, bis auf semantisch äquivalente Situationen.
		\item[M600] Eine einfache KI kann auf Basis von Gewichten einen Zug auswählen.
		\item[M700] Die KI kann ein- und ausgeschaltet werden.
		\item[M800] Um die KI zu trainieren können durch veschiedene Belohnungsstrategien die Gewichte der Pfade verändert werden.
	\end{itemize}
	%Tic-Tac-Toe-Brett und Regeln
	%History-Stack
	%Visualisierung der nachfolgenden Konfigurationen
	%Visualisierung einer einfachen KI (Elimination)
	%Block2
	%Ein- und Ausschalten von KI-Zügen
	%Spielereinfluss auf Gewichte?
\subsection{Soll-Kriterien}
	\begin{itemize}
		\item[S100]
		\item[S200]
	\end{itemize}

\subsection{Kann-Kriterien}
	\begin{itemize}
		\item[K100]
		\item[K200]
	\end{itemize}
	%%%Spätestens hier beginnt Wunschdenken
	%Auswahl mehrerer KIs
	%Implementierung von komplexerer KI (gewichtet Pfad)
	%Implementierung von fehlerloser KI
	%automatisches Trainieren
	%%%Absolutes Wunschdenken
	%Interface für andere Spiele als Tic-Tac-Toe anbieten
	
\subsection{Abgrenzungskriterien}
	\begin{itemize}
		\item[A100] Die Gewichte der KI können nicht manuell angepasst werden
	\end{itemize}
\section{Definition of Done}%Checkliste zur Überprüfung, ob ein Increment fertig ist
\begin{itemize}
	\item Unsere UnitTests laufen alle durch
	\item 
\end{itemize}
%%%Beispiele
%UnitTests laufen
%run dev läuft
%Frage: Haben wir einen Ablauf um Funktionalität zu testen / ob oder welche Ziele erreicht wurden?
%Idee: Codeänderungen nur auf Branches pushen, jemand anderes macht Review und mergt dann
%Gibt es ein Konzept wann Tests geschrieben werden?
%Trello-Todos sind fertig
\section{Definition of Fun}%Vereinbarungen über Zusammenarbeit
tba
%Probleme nicht verstecken
%Pünktlichkeit bei Treffen?
%Abgrenzung zu DoD ist mir nicht klar
%Vielleicht kann man hier einen typischen Arbeitsablauf skizzieren?
%Wollen wir commits im Discord erwähnen?
%Was erwarten wir, wie oft die anderen im Discord lesen?
\section{Wichtige Entscheidungen}
\begin{itemize}
\item Ein 3x3-Spielbrett wird auf einer HTML und Javascipt basierten Webseite angezeigt und auf diesem ist Tic-Tac-Toe spielbar.
\end{itemize}
%Ich glaube hier gibt es noch nicht so richtig was
\section{Projektplanung}%Verteilung der Scrum-Master-Tätigkeit
\begin{tabular}{lc}
	Sprint & Scrum-Master\\
	Entwurf & \\
	Quali-sicherung & \\
	Abschluss & \\
\end{tabular}
\end{document}